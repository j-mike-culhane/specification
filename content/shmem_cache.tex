\apisummary{
    \FUNC{shmem\_cache} controls data cache utilities.
}

\begin{apidefinition}

\begin{DeprecateBlock}
\begin{Csynopsis}
void shmem_clear_cache_inv(void);
void shmem_set_cache_inv(void);
void shmem_clear_cache_line_inv(void *dest);
void shmem_set_cache_line_inv(void *dest);
void shmem_udcflush(void);
void shmem_udcflush_line(void *dest);
\end{Csynopsis}
\end{DeprecateBlock}

\begin{DeprecateBlock}
\begin{Fsynopsis}
CALL SHMEM_CLEAR_CACHE_INV
CALL SHMEM_SET_CACHE_INV
CALL SHMEM_SET_CACHE_LINE_INV(dest)
CALL SHMEM_UDCFLUSH
CALL SHMEM_UDCFLUSH_LINE(dest)
\end{Fsynopsis}
\end{DeprecateBlock}

\begin{apiarguments}

\apiargument{IN}{dest}{A data object that is local to the \ac{PE}.  \VAR{dest}
    can be of any noncharacter type. When using \Fortran, \VAR{dest} can be of any
    kind.}

\end{apiarguments}

\apidescription{   
    \FUNC{shmem\_set\_cache\_inv} enables automatic cache coherency mode.
    
    \FUNC{shmem\_set\_cache\_line\_inv} enables automatic cache coherency mode for
    the cache line associated with the address of \VAR{dest} only.
    
    \FUNC{shmem\_clear\_cache\_inv} disables automatic cache coherency mode
    previously enabled by \FUNC{shmem\_set\_cache\ \_inv} or
    \FUNC{shmem\_set\_cache\_line\_inv}.
    
    \FUNC{shmem\_udcflush} makes the entire user data cache coherent.
    
    \FUNC{shmem\_udcflush\_line} makes coherent the cache line that corresponds with
    the address specified by \VAR{dest}.
}

\apireturnvalues{
    None.
}

\apinotes{
    These routines have been retained for improved backward compatibility with
    legacy architectures.  They are not required to be supported by implementing
    them as \VAR{no-ops} and where used, they may have no effect on cache line
    states.
}

\begin{apiexamples}

None.

\end{apiexamples}

\end{apidefinition}
