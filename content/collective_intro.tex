\emph{Collective routines} are defined as communication or synchronization
operations on a group of \acp{PE} called an active set. The collective
routines require all \acp{PE} in the active set to simultaneously call the
routine.  A \ac{PE} that is not in the active set calling the collective
routine results in undefined behavior.  All collective routines have an
active set as an input parameter except \FUNC{shmem\_barrier\_all} and
\FUNC{shmem\_sync\_all}. Both \FUNC{shmem\_barrier\_all} and
\FUNC{shmem\_sync\_all} must be called by all \acp{PE} of the \openshmem program.

The active set is defined by the arguments \VAR{PE\_start}, \VAR{logPE\_stride},
and \VAR{PE\_size}.  \VAR{PE\_start} specifies the starting \ac{PE} number and
is the lowest numbered PE in the active set.  The stride between successive
\acp{PE} in the active set is $2^{logPE\_stride}$ and \VAR{logPE\_stride} must
be greater than or equal to zero.  \VAR{PE\_size} specifies the number of
\acp{PE} in the active set and must be greater than zero.  The active set must
satisfy the requirement that its last member corresponds to a valid \ac{PE}
number, that is
$0 \le PE\_start + (PE\_size - 1) * 2^{logPE\_stride} < npes$.
All \acp{PE} participating in the collective routine must provide the same
values for these arguments.  If any of these requirements are not met, the
behavior is undefined.

Another argument important to collective routines is \VAR{pSync}, which is a
symmetric work array.  All \acp{PE} participating in a collective must pass the
same \VAR{pSync} array.  On completion of a collective call, the \VAR{pSync} is
restored to its original contents.  The user is permitted to reuse a \VAR{pSync}
array if all previous collective routines using the \VAR{pSync} array have been
completed by all participating \acp{PE}.  One can use a synchronization
collective routine such as \FUNC{shmem\_barrier} to ensure completion of previous collective
routines. The \FUNC{shmem\_barrier} and \FUNC{shmem\_sync} routines allow the same
\VAR{pSync} array to be used on consecutive calls as long as the \acp{PE}
in the active set do not change.

All collective routines defined in the Specification are blocking.  The
collective routines return on completion.  The collective routines defined in
the \openshmem Specification are:

\begin{itemize}
\item \FUNC{shmem\_barrier\_all}
\item \FUNC{shmem\_barrier}
\item \FUNC{shmem\_sync\_all}
\item \FUNC{shmem\_sync}
\item \FUNC{shmem\_broadcast\{32, 64\}}
\item \FUNC{shmem\_collect\{32, 64\}}
\item \FUNC{shmem\_fcollect\{32, 64\}}
\item Reductions for the following operations: AND, MAX, MIN, SUM, PROD, OR, XOR
\item \FUNC{shmem\_alltoall\{32, 64\}}
\item \FUNC{shmem\_alltoalls\{32, 64\}}
\end{itemize}
