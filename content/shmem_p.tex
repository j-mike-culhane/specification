\apisummary{
    Copies one data item to a remote \ac{PE}.
}

\begin{apidefinition}

\begin{C11synopsis}
void shmem_p(TYPE *dest, TYPE value, int pe);
void shmem_p(shmem_ctx_t ctx, TYPE *dest, TYPE value, int pe);
\end{C11synopsis}
where \TYPE{} is one of the standard \ac{RMA} types specified by Table \ref{stdrmatypes}.

\begin{Csynopsis}
void shmem_<TYPENAME>_p(TYPE *dest, TYPE value, int pe);
void shmem_ctx_<TYPENAME>_p(shmem_ctx_t ctx, TYPE *dest, TYPE value, int pe);
\end{Csynopsis}
where \TYPE{} is one of the standard \ac{RMA} types and has a corresponding \TYPENAME{} specified by Table \ref{stdrmatypes}.

\begin{apiarguments}
    \apiargument{IN}{ctx}{Context on which to perform the operation.  When this
    argument is not provided, \CONST{SHMEM\_CTX\_DEFAULT} is used.}
    \apiargument{IN}{addr}{The remotely accessible array element or scalar data object
    which will receive the data on the remote \ac{PE}.}
    \apiargument{IN}{value}{The value to be transferred to \VAR{addr} on the
    remote \ac{PE}.}
    \apiargument{IN}{pe}{The number of the remote \ac{PE}.}
\end{apiarguments}

\apidescription{
    These routines provide a very low latency put capability for single elements of
    most basic types.
    
    As with \FUNC{shmem\_put}, these routines start the remote transfer and may
    return before the data is delivered to the remote \ac{PE}.  Use
    \FUNC{shmem\_quiet} to force completion of all remote \PUT{} transfers.
}

\apireturnvalues{
    None.
}

\apinotes{
    None.
}

\begin{apiexamples}

    \apicexample
    {The following example uses \FUNC{shmem\_p} in a \Cstd[11] program.}
    {./example_code/shmem_p_example.c}
    {}

\end{apiexamples}

\end{apidefinition}
