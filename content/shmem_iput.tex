\apisummary{
    Copies strided data to a specified \ac{PE}.
}

\begin{apidefinition}

\begin{C11synopsis}
void @\FuncDecl{shmem\_iput}@(TYPE *dest, const TYPE *source, ptrdiff_t dst, ptrdiff_t sst, size_t nelems, int pe);
void @\FuncDecl{shmem\_iput}@(shmem_ctx_t ctx, TYPE *dest, const TYPE *source, ptrdiff_t dst, ptrdiff_t sst, size_t nelems, int pe);
\end{C11synopsis}
where \TYPE{} is one of the standard \ac{RMA} types specified by Table \ref{stdrmatypes}.

\begin{Csynopsis}
void @\FuncDecl{shmem\_\FuncParam{TYPENAME}\_iput}@(TYPE *dest, const TYPE *source, ptrdiff_t dst, ptrdiff_t sst, size_t nelems, int pe);
void @\FuncDecl{shmem\_ctx\_\FuncParam{TYPENAME}\_iput}@(shmem_ctx_t ctx, TYPE *dest, const TYPE *source, ptrdiff_t dst, ptrdiff_t sst, size_t nelems, int pe);
\end{Csynopsis}
where \TYPE{} is one of the standard \ac{RMA} types and has a corresponding \TYPENAME{} specified by Table \ref{stdrmatypes}.

\begin{CsynopsisCol}
void @\FuncDecl{shmem\_iput\FuncParam{SIZE}}@(void *dest, const void *source, ptrdiff_t dst, ptrdiff_t sst, size_t nelems, int pe);
void @\FuncDecl{shmem\_ctx\_iput\FuncParam{SIZE}}@(shmem_ctx_t ctx, void *dest, const void *source, ptrdiff_t dst, ptrdiff_t sst, size_t nelems, int pe);
\end{CsynopsisCol}
where \SIZE{} is one of \CONST{8, 16, 32, 64, 128}.

\begin{Fsynopsis}
INTEGER dst, sst, nelems, pe
CALL @\FuncDecl{SHMEM\_COMPLEX\_IPUT}@(dest, source, dst, sst, nelems, pe)
CALL @\FuncDecl{SHMEM\_DOUBLE\_IPUT}@(dest, source, dst, sst, nelems, pe)
CALL @\FuncDecl{SHMEM\_INTEGER\_IPUT}@(dest, source, dst, sst, nelems, pe)
CALL @\FuncDecl{SHMEM\_IPUT4}@(dest, source, dst, sst, nelems, pe)
CALL @\FuncDecl{SHMEM\_IPUT8}@(dest, source, dst, sst, nelems, pe)
CALL @\FuncDecl{SHMEM\_IPUT32}@(dest, source, dst, sst, nelems, pe)
CALL @\FuncDecl{SHMEM\_IPUT64}@(dest, source, dst, sst, nelems, pe)
CALL @\FuncDecl{SHMEM\_IPUT128}@(dest, source, dst, sst, nelems, pe)
CALL @\FuncDecl{SHMEM\_LOGICAL\_IPUT}@(dest, source, dst, sst, nelems, pe)
CALL @\FuncDecl{SHMEM\_REAL\_IPUT}@(dest, source, dst, sst, nelems, pe)
\end{Fsynopsis}

\begin{apiarguments}
    \apiargument{IN}{ctx}{\oldtext{The context on which to perform the operation.} \newtext{A context handle specifying the context on which to perform the operation.}
        When this argument is not provided, the operation is performed on
        \oldtext{\CONST{SHMEM\_CTX\_DEFAULT}} \newtext{the default context}.}
    \apiargument{OUT}{dest}{Array to be updated on the remote \ac{PE}. This data
        object  must be remotely accessible.}
    \apiargument{IN}{source}{Array containing the data to be copied.}
    \apiargument{IN}{dst}{The stride between consecutive elements of the \dest{}
        array.  The stride is scaled by the element size of the \dest{} array.  A
        value of \CONST{1} indicates contiguous data.  \VAR{dst} must be of type
        \CTYPE{ptrdiff\_t}.  When using \Fortran, it must be a default integer value.}
    \apiargument{IN}{sst}{The  stride between consecutive elements of the
        \source{} array.  The stride is scaled by the element size of the \source{}
        array.  A  value of \CONST{1} indicates contiguous data.  \VAR{sst} must be
        of type \CTYPE{ptrdiff\_t}.  When using \Fortran, it must be a
        default integer value.}
    \apiargument{IN}{nelems}{Number of elements in the \dest{} and \source{}
        arrays.  \VAR{nelems} must be of type \VAR{size\_t} for \Cstd.  When
        using \Fortran, it must be  a constant, variable, or array element of
        default integer type.}
    \apiargument{IN}{pe}{\ac{PE} number of the remote \ac{PE}.  \VAR{pe} must be
        of type integer.   When using  \Fortran, it must be a constant,
        variable, or array element of default integer type.}
\end{apiarguments}


\apidescription{
    The \FUNC{iput} routines provide a method  for  copying strided data
    elements (specified by \VAR{sst}) of an array from a \source{} array on the
    local \ac{PE} to locations specified by stride \VAR{dst} on a \dest{} array
    on specified remote \ac{PE}. Both strides, \VAR{dst} and \VAR{sst}, must be
    greater than or equal to \CONST{1}. The routines return when the data has
    been copied out of the \VAR{source} array on the local \ac{PE} but not
    necessarily before the data has been delivered to the remote data object.
    \newtext{
    If the context handle \VAR{ctx} does not correspond to a valid context,
    the behavior is undefined.
    }
}

\apidesctable{
    The \dest{} and \source{} data objects must conform to typing constraints,
    which are as follows:
}{Routine}{Data type of \VAR{dest} and \VAR{source}}
    \apitablerow{shmem\_iput4, shmem\_iput32}{Any noncharacter type
        that has a storage size equal to \CONST{32} bits.}
    \apitablerow{shmem\_iput8}{\Cstd: Any noncharacter type that
        has a storage size equal to \CONST{8} bits.}
    \apitablerow{}{\Fortran: Any noncharacter type that
        has a storage size equal to \CONST{64} bits.}
    \apitablerow{shmem\_iput64}{Any noncharacter type that
        has a storage size equal to \CONST{64} bits.}
    \apitablerow{shmem\_iput128}{Any noncharacter type that has a
        storage size equal to \CONST{128} bits.}
    \apitablerow{SHMEM\_COMPLEX\_IPUT}{Elements of type complex of default size.}
    \apitablerow{SHMEM\_DOUBLE\_IPUT}{Elements of type double precision.}
    \apitablerow{SHMEM\_INTEGER\_IPUT}{Elements of type integer.}
    \apitablerow{SHMEM\_LOGICAL\_IPUT}{Elements of type logical.}
    \apitablerow{SHMEM\_REAL\_IPUT}{Elements of type real.}

\apireturnvalues{
    None.
}

\apinotes{
    When using \Fortran, data types must be of default size.  For example, a
    real variable must be declared as  \CONST{REAL}, \CONST{REAL*4} or
    \CONST{REAL(KIND=KIND(1.0))}.
    See Section \ref{subsec:memory_model} for a definition of the term
    remotely accessible.
}

\begin{apiexamples}

\apicexample
    {Consider the following \FUNC{shmem\_iput} example for \Cstd[11] programs.}
    {./example_code/shmem_iput_example.c}
    {}
\end{apiexamples}

\end{apidefinition}
